\documentclass{article}

\usepackage{amsmath}
\usepackage{amssymb}
\usepackage{bm}
\usepackage{cite}
\usepackage{graphicx}
\usepackage{enumerate}
\usepackage{url}
\usepackage{mathrsfs}
\usepackage{eucal}
\usepackage{color}
\usepackage[hidelinks]{hyperref}

\usepackage[amsmath,thmmarks]{ntheorem}
\newtheorem{lemma}{Lemma}
\newtheorem{theorem}{Theorem}
\theoremstyle{definition}
\newtheorem*{exercise}{Exercise}
\newtheorem*{remark}{Remark}
\newtheorem*{example}{Example}

\newcommand\NB[1]{{\color{red}(NB: #1)}}

\title{Formal Verification and Code-Generation of the Mersenne-Twister Algorithm}
\author{Takafumi Saikawa \and Kensaku Tanaka}

\begin{document}

\maketitle

\begin{abstract}
\NB{We want to advertize that this is the first formalization of
  Mersenne-Twister. The literature needs to be checked.}
\end{abstract}

\section{Introduction}

\section{Formalization of the Algorithm}
One based on binary arithmetic : Ruby to OCaml to Coq.
Another which is more algebraic : from the paper\cite{matsumoto1998acm}

\section{Code Generation}
Fairly easy using the codegen framework\cite{tanaka2018jip}

\NB{Speed comparison between implementations?}

\section{Related Work}

\section{Conclusions and Future Work}

% \appendix
% 
% \section{Detailed Proofs and Technicalities?}

\bibliographystyle{plain}
\bibliography{bib}

\end{document}